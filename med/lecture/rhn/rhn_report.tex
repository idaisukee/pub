\documentclass[10pt,uplatex]{jsarticle}
\usepackage[jis2004]{otf}
\usepackage[top=5truemm,bottom=20truemm,left=10truemm,right=10truemm]{geometry}
\begin{document}
\section*{耳鼻咽喉科学 report}

\begin{flushright}
2016 年 8 月 9 日 02113072 井上 大輔
\end{flushright}

\section{顔面神経麻痺の原因}

顔面神経麻痺は,両側性のものと片側性のものがある.片側性のものは,さらに,中枢性と末梢性に別れる.中枢性顔面神経麻痺は,顔面神経核より上位の障害によるものである.末梢性顔面神経麻痺は,顔面神経核より下位の障害によるものである.

顔面神経麻痺の原因となる障害は,部位により,頭蓋内障害・側頭骨内障害・頭蓋外障害に分類される.

頭蓋内障害には,小脳橋核部の炎症・腫瘍・循環障害がある.

側頭骨内障害は,内耳道・顔面神経管内の障害である.特発性のものが最も多い.耳 herpes に伴うもの ( Hunt 症候群 ) が次に多い.ほか,中耳炎,殊に中耳真珠腫症,内耳道内の聴神経腫瘍,顔面神経特発の神経鞘腫,側頭骨に発する癌腫,外傷による側頭骨骨折,耳手術の副損傷が,側頭骨内障害に含まれる.側頭骨骨折の原因には,交通事故,建設現場での労働災害が多い.

頭蓋外障害には,耳下腺腫瘍,特に悪性腫瘍が多い.ほか,耳介やその周囲の腫瘍・外傷・手術等がある.

参考文献 [0] の pp. 20 - 1 を参照した.

\section{Button 型電池を乳児が飲み込んでしまったときの対応}

まず,患者体内での button 型電池の位置を調べる.

電池が食道内にあるときは,直ちに,先端に磁石を装着した tube ( magnet tube ) を用いて摘出する.短時間で糜爛・穿孔するおそれがあるため,直ちに処置する.

胃内にあるときも, magnet tube により透視下で摘出する.もし magnet tube がないとき,摘出できないとき, 12 時間以上胃内に停滞するならば全身麻酔下に内視鏡で摘出する.

幽門を越えているときは,数日おきに腹部単純 X 線撮影により位置を調べる.同じ場所に留まっていないならば,排泄を確認するまで観察を続ける.

参考文献 [3] の p. 67 を参照した.

\section{伝音難聴と感音難聴}

難聴は,伝音難聴・感音難聴および両者の合併する混合難聴に分類される.

\subsection{伝音難聴}

伝音難聴は,外耳・中耳のいずれかまたは両方が冒され,音または振動の干渉によって起きる聴覚の障害を言う.気導聴力が障害され,骨導聴力は正常である.このため,純音聴力検査で骨導気導差が観測される.

原因を挙げる.外耳の疾患としては耳垢塞栓・外耳道狭窄・外耳道閉鎖症がある.中耳の疾患としては耳小骨奇形等の中耳奇形,耳管狭窄症・鼓膜裂傷・急性および慢性滲出性中耳炎等の各種中耳炎,耳硬化症,中耳腫瘍等がある.

治療には,鼓室形成術・人工中耳 ( 全植込み型補聴器・半植込み型補聴器 ) 装用がある.治療により聴力が改善することが多い.

参考文献 [1] の p. 1952 を参照した.

\subsection{感音難聴}

感音難聴は,音の分析総合機構の障害による難聴である.

迷路性 ( 内耳性 ) 難聴と後迷路性 ( 中枢性 ) 難聴に分類される.

迷路性難聴はさらに,内耳基底回転の障害によるものと内耳全体の障害によるものに分類される.基底回転は蝸牛窓・前庭窓に近いため炎症を生じやすい.また,栄養血管が鋭角に曲がって分布する.さらに,全ての波長の音に反応してゆり動かされる.これらの理由により,基底回転の障害が多い.基底回転は高音域を担当するため,この部位の障害は高音域の聴力の低下をもたらす.

参考文献 [2] の p. 92 を参照した.

\section{北海道と本州の花粉症}

花粉症とは, Allergie 性鼻炎のうち,花粉抗原によって発症するものを言う.

花粉症の原因となる植物のうち主要なものは,ハンノキ ( 榛の木 ) 属・スギ ( 杉 ) ・ヒノキ ( 檜 ) 科・シラカンバ ( 白樺 ) ・イネ ( 稲 ) 科・ブタクサ ( 豚草 ) 属・ヨモギ ( 蓬 ) 属・カナムグラ ( 鉄葎 ) である.地域によって主要な植物とその開花時期が異なるので,診断の手がかりになる.

北海道では,シラカンバの開花期間が長く,単位時間あたりの花粉放出量も多い.スギ・ブタクサ属も 4 月上旬に開花するが,開花期間は本州の 3 割程度である.従って,シラカンバによる花粉症が北海道では重要である.

本州では,スギ・ヒノキ科・イネ科・ブタクサ属の開花期間が長く,単位時間あたりの花粉放出量が大きい.従って,本州ではこれらの植物による花粉症が重要である.なお,本州のなかでもまた地域によって主要な植物とその開花時期が異なるので,注意が必要である.

参考文献 [2] の pp. 305, 309 を参照した.

\section{嚥下の mechanism}

嚥下運動は意識的に開始され,不随意的反射運動に以降する.食塊移動からは口腔相・咽頭相・食道相の 3 相に分割される.神経機構からは口腔期・咽頭期・食道期の 3 期に分割される.正常嚥下では相・期は一致するが,他方,嚥下障害例では一致しない.

\subsection{口腔相}

随意運動が起きる.舌の上に集められた食塊が,口腔の閉鎖,頬筋の収縮による頬の圧迫,顎舌骨筋や縦舌筋の収縮による舌の挙上と口腔内圧の上昇により咽頭腔に送られる.食塊が口峡を通過した瞬間に,次の相に入る.

\subsection{咽頭相}

不随意運動が起きる.口蓋舌・口蓋咽頭筋の収縮による口峡の縮小,舌背の後退挙上による口腔と咽頭腔の遮断,軟口蓋挙上筋の収縮による口蓋挙上と Passavant 隆起によって,上咽頭・中咽頭が遮断される.同時に舌・舌骨も引き上げられ,喉頭蓋基部も喉頭とともに挙上しながら喉頭腔を覆い,かつ,喉頭筋の作用によって声門も閉じて咽頭と気管も遮断される.咽頭収縮筋が収縮して咽頭腔が狭められ,食塊は食道に押し進められる.食道前壁は厚い筋膜層により輪状軟骨の後面に付着しているので,喉頭が挙上されると下咽頭と梨状陥凹が開かれ,陰圧が生じる.この陰圧は嚥下に寄与する.

\subsection{食道相}

食塊が咽頭壁の蠕動運動により押し上げられるとともに,食道腔内に生じた陰圧に吸引されて食道に入る.

\vspace{1em}

参考文献 [2] の pp. 395 - 6 を参照した.

\section{良性発作性頭位めまい症}

末梢性めまいの中で最も高頻度の疾患である.症状の特徴は,

\begin{itemize}
\item 特定の頭位 ( めまい頭位 ) で回転性めまいが起きること
\item めまいが,めまい頭位で次第に増強し,次いで減弱・消失すること
\item 繰り返しめまい頭位をとると,めまいは軽くなるか消失すること
\item 難聴・耳鳴といった蝸牛症状がないこと
\end{itemize}

である.めまいの程度は激しいが,数十秒以内に減弱・消失する例が多い.

頭位変換眼振検査を行うと,坐位から懸垂頭位へ,あるいはその逆へ頭位を変換したときに,回旋方向が時計方向から反時計方向,あるいはその逆へと眼振方向が変化する.

薬物療法・理学療法が有効である.予後は一般に良好である.

参考文献 [1] の p. 2893 を参照した.

\section{上咽頭癌・中咽頭癌・下咽頭癌の risk factor}

\subsection{上咽頭癌}

参考文献 [1] の p. 1301 によると,東南 Asia に多く,男女比は 1:2 であり, 40 - 60 歳に多い.参考文献 [0] の p.103 によると, EB virus の関与が考えられている.

\subsection{中咽頭癌}

男性に多い.喉頭・口腔・食道等との重複癌が多いため,外的因子の関与が示唆される.

参考文献 [1] の p.1840 を参照した.

\subsection{下咽頭癌}

男性では,度数の高い酒の飲酒習慣が risk factor となる.

女性では,鉄欠乏性貧血を伴う Plummer - Vinson 症候群,頸部への放射線照射が risk factor となる. 60 歳代の男性に多い.ただし,発生部位別に見ると,梨状陥凹癌は男性に多く,輪状後部癌は女性に多く,後壁癌は性差がない.

参考文献 [1] の pp. 379 - 80 を参照した.

\section*{参考文献}


[0] 鈴木淳一ほか 2007 『標準耳鼻咽喉科・頭頸部外科学』第 3 版第 8 刷

[1] 伊藤正男ほか 2009 『医学書院 医学大辞典』第 2 版第 1 刷

[2] 加我君孝ほか 2013 『新耳鼻咽喉科学』第 11 版第 1 刷

[3] 内山聖ほか 2013 『標準小児科学』第 8 版第 1 刷

\end{document}
