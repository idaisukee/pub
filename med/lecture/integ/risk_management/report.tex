\documentclass[11pt,dvipdfmx,uplatex]{jsarticle}
\usepackage[jis2004]{otf}
\usepackage[]{tabularx}
\usepackage[para]{footmisc}
\usepackage[top=10truemm,bottom=20truemm,left=15truemm,right=15truemm]{geometry}
\usepackage{booktabs}
\usepackage[hiresbb]{graphicx}
\usepackage{siunitx}

\newcommand{\liter}{\ell}
\newcommand{\imagewidth}{.3\textwidth}
\begin{document}

2018 年 7 月 16 日

02113072 井上大輔

\section*{医療事故と risk management}

講義では 1999 年の横浜市立大学医学部附属病院の医療事故が扱われた.これについて調べた.

\section{事故の概要}

1999 年 1 月 11 日,同院で A 氏 ( 74 歳男性 ) の心臓疾患に対する僧帽弁形成術または弁置換術と, B 氏 ( 84 歳男性 ) の肺疾患に対する試験開胸術中生検 ( 悪性の場合右上葉切除 ) が予定された.ところが,

\begin{quote}
    1人の病棟看護婦が2人の患者を同時に手術室に移送した。

    手術室交換ホールでの患者受け渡し時に患者を取り違えた。

    患者とカルテを別々の窓口で引き渡し,別々に手術室に移送した。

    患者への名前の呼びかけと患者の返事が,患者を識別する方法とはなり得なかった。

    患者A氏の背中に貼ってあったフランドルテープが患者識別につながらなかった。また,申し送りも活かされなかった。

    麻酔開始前から主治医が患者に立ち会っておらず,患者の識別を行っていなかった。

    患者の歯の状況や頭髪の様子の違い(長さ,色)によって患者の取り違えに気づかなかった。

    B氏の麻酔準備から開胸前の間に実施した各種の検査結果が,術前の検査結果と異なることに疑問を持ち,一応の確認はしたものの,患者の識別には至らなかった。

    開胸後も,患者の取り違えに気づかずに手術を続行した。 

\end{quote}

という事態が発生した $^1$.

\section{原因の分析}

横浜市立大学医学部附属病院の医療事故に関する事故対策委員会は,この事故の原因として次の各項があったと分析している:

\begin{quote}
	1 病院運営システム上の問題点について

	1.1 患者の取り違えを起こしかねない患者移送,引き継ぎの運用システムであった。

	1.2 患者確認の手順,方法が決められていなかった。

	1.3 事故が起こり得ることを想定しておらず,二重,三重の安全策,危機管理の方策がなされていなかった。

	1.4 手術室での種々の疑問点を統合的,横断的に把握する機能が働かなかった。

	2 病院組織管理上の問題

	2.1 手術に際して「患者確認」という患者の安全と人権にかかわる基本的事項について,関係各部(看護部,麻酔科,外科),また病院全体としての指導・教育が不十分であった。

	2.2 患者の安全確保のため,主治医や担当医として本来行うべき役割と責任の範囲が不明確であり,指導管理が関係各科で不十分であった。

	2.3 病院全体として医療事故を予防し,また事故が起こり得ることを想定した対策,訓練などを行う管理体制の整備が不十分であった。

	3 教育上の問題点

	3.1  高度に専門分化,複雑化する医療を患者の立場に立ち,見直す教育,訓練がされていなかった。
	
\end{quote}

\section{事故後の対策}

同院は事故後,以下の対策を実施した:

\begin{quote}
1 病棟の看護婦(士)が患者を手術室に移送するのは,1回に1人を徹底した。

2 定時手術の場合は,病棟から手術室への患者移送に際し,主治医の1人が手術室交換ホールまで付き添い,手術室の看護婦(士)に患者が引き受けられたことを確認する。

3 患者とカルテ類とが離れることのないよう,患者,カルテ類ともにハッチウェイで引き継ぎ,一緒に手術室に移送する。

4 入院患者は,入院時に手首等に氏名,年齢などが記入された患者識別バンドをつける。

5 入院患者の手術等の際には,病棟看護婦(士)立会いのもとに患者の足底にマジックで氏名を書く。

6 手術室交換ホールにおいて,手術室看護婦(士)が行う病棟看護婦(士)からの患者引き受け時の確認は,次のとおりとする。

    患者の氏名を患者の言葉で確認する。

    患者識別バンドのID番号,氏名,年齢,性別,入院月日をカルテと照合し確認する。

    足底に書かれた患者の氏名をカルテと照合し確認する。 

7 麻酔科医は患者を術前訪問し,当日はハッチウェイで出迎え,手術室の看護婦(士)とともに患者確認を行う。

8 手術室内では,麻酔科医は主治医とともにカルテの血液型検査結果のID番号,氏名を患者識別バンドと照合・確認し,患者識別バンドに血液型を記入する。

9 麻酔科医は,主治医とともに患者確認を行った後において麻酔を開始する。

10 手術室の看護婦(士)による術前の患者訪問は,原則として手術に立ち会う看護婦(士)が行い,訪問時には,患者確認の一助となるよう患者の外見的・身体的な特徴についても記録する。担当以外の看護婦(士)が訪問した場合であっても,術前訪問記録用紙に患者の特徴などを記入し,担当の看護婦(士)に引き継ぐ。

11 麻酔科医,執刀医は患者の確認に疑問を持った場合には,その疑問を解決するまでは新しい段階に進まない。

12 外科医が行うべき患者確認方法を明確にした。

13 麻酔科医が行うべき患者確認方法を明確にした。

14 看護部は,「手術室移送時の手順」や「手術室入室時の手順」(マニュアル)に事故対策委員会の決定事項を明記した。

15 手術部専任医が行う患者の安全管理に対する役割を明確にした。
\end{quote}

WHO は,「チームは、正しい患者の正しい部位に手術を行う」という目標を定め,このために「確認」「マーキング」「タイムアウト」の 3 手順を定めている.確認は,「手術の決定が行われてから患者が手術を受けるまでの全ての段階で、患者、部位と処置が正しいことを確認することである」と定義している.具体的には,手術施設入院時,患者のケアの責任が他のスタッフに変わる時点,患者が術前区域を離れる前,手術室入室前に「の手順は可能な限り、患者が覚醒し、意識が清明な状
況で患者とともに行われる。確認は、説明の間に患者に表
示を行い、患者を特定することで行われる。部位や左右、
予定術式は患者診療録と画像をチェックすることで確認す
る。これは、患者ケアに関与する全てのチームメンバーが
関わるべき積極的なプロセスである。チームメンバー全員
が確認に関わるが、それぞれのチェックは独立的に行うこ
と」と定めている $^2$.


\section*{参考文献}

1 横浜市立大学医学部附属病院の医療事故に関する事故対策委員会.  横浜市立大学医学部附属病院の医療事故に関する中間とりまとめ ( 2018 年 7 月 15 日閲覧 )

2 世界保健機構. WHO 安全な手術のためのガイドライン 2009
\end{document}
