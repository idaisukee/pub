\documentclass[11pt,dvipdfmx,uplatex]{jsarticle}
\usepackage[jis2004]{otf}
\usepackage[]{tabularx}
\usepackage[para]{footmisc}
\usepackage[top=10truemm,bottom=20truemm,left=15truemm,right=15truemm]{geometry}
\usepackage{booktabs}
\usepackage[hiresbb]{graphicx}
\usepackage{siunitx}

\newcommand{\liter}{\ell}
\newcommand{\imagewidth}{.3\textwidth}
\begin{document}
2018 年 7 月 16 日

02113072 井上大輔
\section*{対応が困難な患者への面接}

講義で benzodiazepine 系薬剤依存が扱われた.これについて調べた.

\section{Benzodiazepine 系薬について}

Benzodiazepine 系薬は,鎮静催眠薬,抗痙攣薬,抗不安薬として使用されている. GABA$_\textrm{A}$ 受容体の GABA とは異なる結合部位に結合し GABA の作用を増強することにより作用を発揮すると考えられている. Benzodiazepine 系薬単独では GABA$_\textrm{A}$ 受容体に内蔵する Cl$-$ channel を開口しないが, benzodiazepine 系薬が存在すると GABA が作用したときのように開口頻度が増加することが観察されている$^1$.

\section{鎮静薬,睡眠薬,または抗不安薬関連障害群}

DSM-5 ( 精神疾患の診断・統計マニュアル ) に,鎮静薬,睡眠薬,または抗不安薬関連障害群が定義されている.この下に,「鎮静薬,睡眠薬,または抗不安薬使用障害」「鎮静薬,睡眠薬,または抗不安薬中毒」「鎮静薬,睡眠薬,または抗不安薬離脱」「他の鎮静薬,睡眠薬,または抗不安薬誘発性障害群」「特定不能の鎮静薬,睡眠薬,または抗不安薬関連障害」の 5 種の疾患が定義されている.

鎮静薬,睡眠薬,または抗不安薬には, benzodiazepine, benzodiazepine 様薬剤 ( 例: zonpidem, zaleplon ), carbamine 酸塩 ( 例: glutethimide, meprobamate ), barbiturate ( 例: secobarbital ), barbiturate 酸様睡眠薬 ( 例: glutethimide, methaqualone ) が含まれる.非 benzodiazepine 系抗不安薬は重大な誤用とは関連しないようであり,この種類には含まれない$^2$.

\section{鎮静薬,睡眠薬,または抗不安薬使用障害}

5 疾患のうち特に「鎮静薬,睡眠薬,または抗不安薬使用障害」について述べる.

診断的特徴は,渇望,欠勤の反復や職務遂行能力の定価,学校の欠席・停学・退学あるいは育児家事の neglect, 中毒の影響をめぐり配偶者と口論あるいは身体的喧嘩が生じてもなお続く物質使用,家族・友人との接触の制限,仕事や学校の回避,趣味, sports, game への参加の低下,物質の反復的使用で機能が障害されたなかでの自動車の運転,または機械の操作である$^2$.

12 ヶ月間における有病率は 12 - 17 歳の人口で 0.3\%, 18 歳以上で 0.2\% と推定されている.成人では男性の有病率が女性のそれよりわずかに高い.しかし 12 - 17 歳では女性の方が高い$^2$.

この疾患は,原発性精神疾患に類似した疾状を呈することがある.呂律の回らない会話,協調運動障害,他の鎮静薬,睡眠薬,または抗不安薬中毒に関連する特徴的病像は,他の医学的疾患 ( 例: 多発性硬化症 ) または以前の頭部外傷 ( 例: 硬膜下血腫 ) の結果であることがあり得る.また, alcohol 使用障害との鑑別も必要である.


\section{日本での添付文書}

Benzodiazepine 系薬の日本での添付文書には,旧来, benzodiazepine 系薬の依存性は依存になりやすい体質の患者に長期間,大量投与した場合のみ出現するとの考え方により,大量連用時の依存性に関する注意喚起が添付文書においてなされてきた.しかし,問題の中核は,濫用や医療外使用によるものではなく,医療上の使用で生じる依存であるとする考えが徐々に浸透してきた.このため,厚生労働省医薬・生活衛生局は,催眠鎮静薬,抗不安薬及び抗てんかん薬のうち,依存性関連の副作用が添付文書に記載されている医薬品について,国内副作用報告の集積状況,依存及び離脱症状に関する文献及び国内ガイドラインに基づき,依存性等の安全性を検討した.この結果,「使用上の注意」改訂を製造販売業者に対して指示することが適切と判断したことから,平成29年3月17日に開催された平成28年度第3回薬事・食品衛生審議会薬事分科会医薬品等安全対策部会での検討後,平成29年3月21日に「使用上の注意」改訂を指示した.改訂では,「認用量の範囲内においても,連用により薬物依存が生じることがあるので, 用量及び使用期間に注意し,慎重に投与すること。 催眠鎮静薬又は抗不安薬として使用する場合には,漫然とした継続投与による長期使用を避けること。投与を継続する場合には,治療上の必要性を検討すること。承認用量の範囲内においても,連用中における投与量の急激な減少又は投与の中止により,原疾患の悪化や離脱症状があらわれることがあるので,投与を中止する場合には,徐々に減量するなど慎重に行うこと。ベンゾジアゼピン受容体作動薬については,統合失調症患者や高齢者に限らず,刺激興奮,錯乱等があらわれることがあるので,観察を十分に行うこと」が新たに添付文書に盛り込まれた.

\section*{参考文献}

1 今井正ら. 標準薬理学. p. 108. 医学書院

2 精神疾患の診断・統計マニュアル. 第 5 版. pp. 543 - 8

3 厚生労働省医薬・生活衛生局. 医薬品・医療機器等安全性情報. No 342

\end{document}
