\documentclass[11pt,dvipdfmx,uplatex]{jsarticle}
\usepackage[jis2004]{otf}
\usepackage[]{tabularx}
\usepackage[para]{footmisc}
\usepackage[top=10truemm,bottom=20truemm,left=15truemm,right=15truemm]{geometry}
\usepackage{booktabs}
\usepackage[hiresbb]{graphicx}
\usepackage{siunitx}

\newcommand{\liter}{\ell}
\newcommand{\imagewidth}{.3\textwidth}
\begin{document}

2018 年 7 月 16 日

02113072 井上大輔
\section*{臨床検査}

講義で 保険医療機関及び保険医療養担当規則が扱われた.これについて調べた.

\section{保険医療機関及び保険医療養担当規則の概要}

保険医療機関及び保険医療養担当規則は,昭和三十二年厚生省令第十五号である.「健康保険法(大正十一年法律第七十号)第四十三条ノ四第一項及び第四十三条ノ六第一項(これらの規定を同法第五十九条ノ二第七項において準用する場合を含む。)の規定に基き、並びに日雇労働者健康保険法(昭和二十八年法律第二百七号)及び船員保険法(昭和十四年法律第七十三号)を実施するため、保険医療機関及び保険医療養担当規則を次のように定める」$^1$という目的で制定されている.


\section{保険診療の禁止事項}

規則は,保険診療の禁止事項を定めている.

第十二条に,「保険医の診療は、一般に医師又は歯科医師として診療の必要があると認められる疾病又は負傷に対して、適確な診断をもととし、患者の健康の保持増進上妥当適切に行われなければならない」と定めている$^1$.これについて厚生労働省北海道厚生局は,「 医師が自ら診察を行わずに治療、投薬(処方せんの交付)、診断書の作成等を行うことは、保険
診療の必要性について医師の判断が的確に行われているとはいえず、保険診療としては認められ
るものではない。
  なお、無診察治療については、保険診療上不適切であるのみならず、医師法違反(「医師は、
自ら診察しないで治療をしてはならない」第20条)に当たるものであり、また、倫理的にも医療
	安全の観点からも極めて不適切な行為であることは言うまでもない」と解釈を示している.さらに,無診察治療の例として,

「定期的に通院する慢性疾患の患者に対し、診察を行わずに投薬。又は、診察を行わずに処方せ
んの交付。 

通院リハビリテーション目的で訪れた患者が、理学療法士によるリハビリテーションを行った
のみで、医師の診察の事実がないのに再診料を請求。 

診療録に、診察に関する記載が全くなかったり、「薬のみ(medication)」等の記載しかない。
(無診察治療の疑い)」

があると述べている$^2$.


第十八条に「保険医は、特殊な療法又は新しい療法等については、厚生労働大臣の定めるもののほか行つてはならない」,第十九条に「保険医は、厚生労働大臣の定める医薬品以外の薬物を患者に施用し、又は処方してはならない。ただし、医薬品、医療機器等の品質、有効性及び安全性の確保等に関する法律(昭和三十五年法律第百四十五号)第二条第十七項に規定する治験(以下「治験」という。)に係る診療において、当該治験の対象とされる薬物を使用する場合その他厚生労働大臣が定める場合においては、この限りでない」,第二十条に「医師である保険医の診療の具体的方針は、前十二条の規定によるほか、次に掲げるところによるものとする」,特に一のハに「健康診断は、療養の給付の対象として行つてはならない」,ほか「投薬は、必要があると認められる場合に行う」「治療上一剤で足りる場合には一剤を投与し、必要があると認められる場合に二剤以上を投与する」「同一の投薬は、みだりに反覆せず、症状の経過に応じて投薬の内容を変更する等の考慮をしなければならない」「投薬を行うに当たつては、医薬品、医療機器等の品質、有効性及び安全性の確保等に関する法律第十四条の四第一項各号に掲げる医薬品(以下「新医薬品等」という。)とその有効成分、分量、用法、用量、効能及び効果が同一性を有する医薬品として、同法第十四条又は第十九条の二の規定による製造販売の承認(以下「承認」という。)がなされたもの(ただし、同法第十四条の四第一項第二号に掲げる医薬品並びに新医薬品等に係る承認を受けている者が、当該承認に係る医薬品と有効成分、分量、用法、用量、効能及び効果が同一であつてその形状、有効成分の含量又は有効成分以外の成分若しくはその含量が異なる医薬品に係る承認を受けている場合における当該医薬品を除く。)(以下「後発医薬品」という。)の使用を考慮するとともに、患者に後発医薬品を選択する機会を提供すること等患者が後発医薬品を選択しやすくするための対応に努めなければならない」「栄養、安静、運動、職場転換その他療養上の注意を行うことにより、治療の効果を挙げることができると認められる場合は、これらに関し指導を行い、みだりに投薬をしてはならない」「投薬量は、予見することができる必要期間に従つたものでなければならないこととし、厚生労働大臣が定める内服薬及び外用薬については当該厚生労働大臣が定める内服薬及び外用薬ごとに一回十四日分、三十日分又は九十日分を限度とする」「注射薬は、患者に療養上必要な事項について適切な注意及び指導を行い、厚生労働大臣の定める注射薬に限り投与することができることとし、その投与量は、症状の経過に応じたものでなければならず、厚生労働大臣が定めるものについては当該厚生労働大臣が定めるものごとに一回十四日分、三十日分又は九十日分を限度とする」「注射は、次に掲げる場合に行う。(1) 経口投与によつて胃腸障害を起すおそれがあるとき、経口投与をすることができないとき、又は経口投与によつては治療の効果を期待することができないとき。(2) 特に迅速な治療の効果を期待する必要があるとき。(3) その他注射によらなければ治療の効果を期待することが困難であるとき」「(注射の) 内服薬との併用は、これによつて著しく治療の効果を挙げることが明らかな場合又は内服薬の投与だけでは治療の効果を期待することが困難である場合に限つて行う」「混合注射は、合理的であると認められる場合に行う」「輸血又は電解質若しくは血液代用剤の補液は、必要があると認められる場合に行う」「手術は、必要があると認められる場合に行う」「処置は、必要の程度において行う」「リハビリテーションは、必要があると認められる場合に行う」「居宅における療養上の管理及び看護は、療養上適切であると認められる場合に行う」「入院の指示は、療養上必要があると認められる場合に行う」「単なる疲労回復、正常分べん又は通院の不便等のための入院の指示は行わない」と定めている.これについて同局は,「医学的評価が十分に確立されていない、「特殊な療法又は新しい療法等」の実施、「厚生労働大臣の定める医薬品以外の薬物」の使用、「研究の目的」による検査の実施などは、保険診療上認められるものではない。(例外)先進医療(高度医療を含む)による一連の診療 ルールに従った治験による薬剤の投与や、これに伴う一連の検査
	
	健康診断の禁止(療担第20条)  健康診断は、保険診療として行ってはならない。
	
濃厚(過剰)診療の禁止(療担第20条)  検査、投薬、注射、手術・処置等は、診療上の必要性を十分考慮した上で、段階を踏んで必要最小限に行う必要がある」と解釈を示している.


\section{特殊療法の禁止について}

第十八条で特殊療法が禁止されている.しかし,「先進医療(高度医療を含む)による一連の診療」「ルールに従った治験による薬剤の投与や、これに伴う一連の検査」は例外となると同局が解釈を示している$^2$.

\subsection{先進医療について}

先進医療について厚生労働省は,「先進医療は、健康保険法等の一部を改正する法律(平成18年法律第83号)において、「厚生労働大臣が定める高度の医療技術を用いた療養その他の療養であって、保険給付の対象とすべきものであるか否かについて、適正な医療の効率的な提供を図る観点から評価を行うことが必要な療養」として、厚生労働大臣が定める「評価療養」の1つとされています。具体的には、有効性及び安全性を確保する観点から、医療技術ごとに一定の施設基準を設定し、施設基準に該当する保険医療機関は届出により保険診療との併用ができることとしたものです。なお、将来的な保険導入のための評価を行うものとして、未だ保険診療の対象に至らない先進的な医療技術等と保険診療との併用を認めたものであり、実施している保険医療機関から定期的に報告を求めることとしています」と述べている$^3$.つまり,先進医療においては条件つきで自費診療と保険診療の併用が許されている.

\seciton{参考文献}

1 保険医療機関及び保険医療養担当規則

2 厚生労働省北海道厚生局. 保険診療の理解のために. https://kouseikyoku.mhlw.go.jp/hokkaido/documents/ika27.pdf . ( 2018 年 7 月 17 日閲覧 )

3 厚生労働省. 先進医療の概要について. https://www.mhlw.go.jp/stf/seisakunitsuite/bunya/kenkou_iryou/iryouhoken/sensiniryo/ . ( 2018 年 7 月 17 日閲覧 )




\end{document}
