\documentclass[11pt,dvipdfmx,uplatex]{jsarticle}
\usepackage[jis2004]{otf}
\usepackage[]{tabularx}
\usepackage[para]{footmisc}
\usepackage[top=10truemm,bottom=20truemm,left=15truemm,right=15truemm]{geometry}
\usepackage{booktabs}
\usepackage[hiresbb]{graphicx}
\usepackage{siunitx}

\newcommand{\liter}{\ell}
\newcommand{\imagewidth}{.3\textwidth}
\begin{document}

2018 年 7 月 16 日

02113072 井上大輔
\section*{診療録記載・退院時要約の作成}

講義で医療事故収集等事業が扱われた.これについて調べた.

\section{事業の概要}

\subsection{事業の目的}

公益財団法人日本医療機能評価機構は,「事業は、医療機関から報告された医療事故情報やヒヤリ・ハット事例を、収集、分析し提供することにより、広く医療機関が医療安全対策に有用な情報を共有するとともに、国民に対して情報を公開することを通じて、医療安全対策の一層の推進を図ることを目的としています。また、医療事故の発生予防・再発防止を促進することを目的に、医療機関や国民に情報を周知するため報告書や医療安全情報を作成し提供しています」と述べている$^1$.

\subsection{参加医療機関}

この事業にはどの医療機関が参加しているのか.

医療事故情報収集・分析・提供事業について同機構は,「医療事故情報収集・分析・提供事業は、法令により医療事故の報告をすることが義務付けられている「報告義務対象医療機関」と、任意でご参加いただいている「参加登録申請医療機関」があります。」と述べている$^1$.報告義務対象医療機関は,医療法施行規則で本事業への医療事故の報告を義務付けられた医療機関であり,具体的には国立研究開発法人及び国立ハンセン病療養所,独立行政法人国立病院機構の開設する病院,学校教育法に基づく大学の附属施設である病院 ( 病院分院を除く),特定機能病院である.特定機能病院について厚生労働省は,「特定機能病院は、高度の医療の提供、高度の医療技術の開発及び高度の医療に関する研修を実施する能力等を備えた病院として、第二次医療法改正において平成5年から制度化され、平成29年6月1日現在で85病院が承認されています」と述べている.北海道では,旭川医科大学病院,札幌医科大学附属病院,北海道大学病院の 3 院が特定機能病院である$^2$.

ヒヤリ・ハット事例収集・分析・提供事業について同機構は,「ヒヤリ・ハット事例収集・分析・提供事業は、全て任意の参加です。発生件数情報のみ報告する方法と、発生件数情報及び事例情報を報告する方法の2種類があります」と述べている$^1$..

医療事故情報収集・分析・提供事業には 1031 機関,ヒヤリ・ハット事例収集・分析・提供事業には 1194 機関が参加している$^1$.

\subsection{報告件数}

2016 年には医療事故情報は 3882 件,ヒヤリ・ハット事例 30318 件報告された$^1$.

\subsection{報告書・年報}

同機構は報告書・年報を発行している.これについて同機構は「本事業では、報告書を四半期に1回、年報を1年に1回、公表しています。/ 報告書では、報告義務対象医療機関からの報告の集計のほか、「分析テーマ」として、報告された事例の詳細
な分析を行っています。分析テーマは、①一般性・普遍性、②発生頻度、③患者への影響度、④防止可能性、⑤
教訓性といった観点から選定します。また、年報では、報告書に掲載していない任意で参加している参加登録申
請医療機関の情報を加えた集計結果や、現地状況確認調査の概要を掲載しています」と述べている.2016 年に取り上げた分析 theme は

\begin{itemize}
\item 腫瘍用薬に関連した事例 ①~④

\item 外観の類似した薬剤の取り違えに関連した事例

\item 人工呼吸器の回路の接続外れに関連した事例

\item 持参薬と院内で処方した薬剤の重複投与に関連した事例

\item 永久気管孔にフィルムドレッシング材を貼付した事例

\item 歯科治療中に異物を誤飲・誤嚥した事例

\item 小児用ベッドからの転落に関連した事例

\item 蘇生時、アドレナリンを投与するところノルアドレナリンを投与した事例

\item 下肢閉塞性動脈硬化症の患者への弾性ストッキング装着に関連した事例
	
\end{itemize}

であった$^1$.

\subsection{事例検索}

同機構は web で事例検索の service を提供している. PDF や CSV で download もできる$^1$.

\section{参考文献}

1 公益財団法人日本医療機能評価機構. http://www.med-safe.jp/pdf/business_pamphlet.pdf ( 2018 年 7 月 16 日閲覧 )

2 厚生労働省. https://www.mhlw.go.jp/stf/seisakunitsuite/bunya/0000137801.html ( 2018 年 7 月 16 日閲覧 )
\end{document}
